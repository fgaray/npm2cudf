\documentclass[letterpaper,12pt]{report}

\usepackage[letterpaper]{geometry}
\usepackage{minted}
\usepackage{url}
\usepackage{tabularx}

\title{Report}
\author{Felipe Garay}

\begin{document}
\maketitle
\newpage
\tableofcontents
\newpage


\chapter{Introduction}




\section{NPM package repository}

The NPM (Node Package Manager) is the de factor package manager for the nodejs
ecosystem. Users use a centralized server called registry where all the public
packages' metadata is stored using a CouchDB database. A public replica of this
database can be found in \url{https://skimdb.npmjs.com/}.

Since CouchDB allows easy replication, we can ask a local database to replicate
the registry of NPM and even to use a continous replication, that means that
every time there is a change in the master database, the replica will be update
with them.

This replication can be accomplish using the interface of CouchDB, pointing the
source to \url{https://skimdb.npmjs.com/} and the destination to ``registry''.

Notice that there is a entry in the database that prevents the replication of
more packages once is copied to the local database. This document can be
identified by the id ``\_design/app'' and it is sufficient to delete it and
resume the replication.

At the moment of the writing of this document the database has a size of 1.3 GB
with 233593	unique documents/packages.

\section{Structure of a NPM package}


NPM packages are stored in the registry in JSON. A typical document for a
package in the database has the following fields and is basically a set of the
different versions of the ``package.json''
\footnote{https://docs.npmjs.com/files/package.json}:

\begin{itemize}
    \item \textbf{\_id}: An unique string id of the package, it is the same that the
        name.
    \item \textbf{name}: The name of the package.
    \item \textbf{description}: A short description of the package.
    \item \textbf{dist-tags}: A JSON dictionary with named tags pointing to versions.
        For example ``latest: 13.0.0''.
    \item \textbf{versions}: A JSON dictionary where each key is a version of the
        package with the following structure:
        \begin{itemize}
            \item \textbf{\_id}: The concatenation of the ``\_id'' of the package
                with the version.
            \item \textbf{name}: The name of the package in this version. It is
                always the same that the name of the package.
            \item  \textbf{version}: Is the same string that in the key of the
                dictionary of ``versions''.
            \item \textbf{description}: The description in this version of the
                package.
            \item \textbf{main}: The Javascript file use as the entry of the program.
                If it is a library, this field is not present.
            \item \textbf{bin}
            \item \textbf{repository}: A JSON dictionary with the information of
                the repository of the proyect:
                \begin{itemize}
                    \item \textbf{type}: A string with the type of the
                        repository. For example ``git''.
                    \item \textbf{url}: The url for cloning the repository.
                \end{itemize}
            \item \textbf{keywords}: A list of keywords to easily find the
                package in the registry.
            \item \textbf{dependencies}: A JSON dictionary where each key if the
                name of the dependency and each value is a semver version (with
                the node's style) or a git URL.
            \item \textbf{devDependencies}: Dependencies needed to run tests or
                in general modify the library.
            \item \textbf{author}: A JSON dictionary with the following keys:
                \begin{itemize}
                    \item \textbf{name}: The name of the author.
                    \item \textbf{email}: The email of the author.
                    \item \textbf{url}: The URL of the webpage of the author
                        (optionl).
                \end{itemize}
            \item \textbf{license}: A name of the license of the code in this
                version.
            \item \textbf{engine}: A list of string where each one is a
                constraint related to the version of node required to run the
                program or library. For example: ``node $>=$0.2.0''.
            \item \textbf{scripts}: A JSON dictionary with scripts to be run in
                the livecycle of the software. The scripts relevante to this
                work are described in the scripts section.
            \item \textbf{dist}: A JSON dictionary with information about the
                source code of the pacakge:
                \begin{itemize}
                    \item \textbf{shasum}: The hash of the file to be
                        downloaded.
                    \item \textbf{tarball}: A tgz file with the code of the
                        version.
                \end{itemize}
            \item \textbf{time}: A dictionary with the timestamp of the
                versions.
            \item \textbf{readme}: The readme to be show to the user in the
                webpage of the NPM registry.
        \end{itemize}
\end{itemize}



\subsection{NPM Scripts}


NPM defines several scripts to be executed in different steps of the instalation
of the software. These are defined in
\footnote{https://docs.npmjs.com/misc/scripts} and the relevant to this work
are:

\begin{itemize}
    \item preinstall
    \item install, postinstall
    \item preuninstall
    \item uninstall
    \item postuninstall
\end{itemize}

A simple map can be made betweet some of this scripts and the one in opam:



\begin{tabularx}{\textwidth}{|c|X|}
        \hline
        NPM script              & opam field    \\ \hline \hline
        preinstall              & build         \\ \hline
        install,postinstall     & install       \\ \hline
        preuninstall,uninstall  & remove        \\ \hline
        \hline
\end{tabularx}



\section{Tools developed}

Several tools were developed to convert the data in the registry into a CUDF
file. Two CouchDB's views, a typescript program and a python program.

The order of execute is: install the views, run deps\_generator.js and finally
npm2cudf.py.

\subsection{Views}



\end{document}
