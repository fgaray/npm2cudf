\documentclass[letterpaper,12pt]{report}

\usepackage[letterpaper]{geometry}
\usepackage{minted}
\usepackage{url}
\usepackage{tabularx}

\title{Report}
\author{Felipe Garay}

\begin{document}
\maketitle
\newpage
\tableofcontents
\newpage


\chapter{Introduction}

\section{NPM package repository}

The NPM (Node Package Manager) is the de factor package manager for the nodejs
ecosystem. Users use a centralized server called registry where all the public
packages' metadata is stored using a CouchDB database. A public replica of this
database can be found in \url{https://skimdb.npmjs.com/}.

Since CouchDB allows easy replication, we can ask a local database to replicate
the registry of NPM and even to use a continous replication, that means that
every time there is a change in the master database, the replica will be update
with them.

This replication can be accomplish using the interface of CouchDB, pointing the
source to \url{https://skimdb.npmjs.com/} and the destination to ``registry''.

Notice that there is a entry in the database that prevents the replication of
more packages once is copied to the local database. This document can be
identified by the id ``\_design/app'' and it is sufficient to delete it and
resume the replication.

At the moment of the writing of this document the database has a size of 1.3 GB
with 233593	unique documents/packages.

\section{Structure of a NPM package}


NPM packages are stored in the registry in JSON. A typical document for a
package in the database has the following fields and is basically a set of the
different versions of the ``package.json''
\footnote{https://docs.npmjs.com/files/package.json}:

\begin{itemize}
    \item \textbf{\_id}: An unique string id of the package, it is the same that the
        name.
    \item \textbf{name}: The name of the package.
    \item \textbf{description}: A short description of the package.
    \item \textbf{dist-tags}: A JSON dictionary with named tags pointing to versions.
        For example ``latest: 13.0.0''.
    \item \textbf{versions}: A JSON dictionary where each key is a version of the
        package with the following structure:
        \begin{itemize}
            \item \textbf{\_id}: The concatenation of the ``\_id'' of the package
                with the version.
            \item \textbf{name}: The name of the package in this version. It is
                always the same that the name of the package.
            \item  \textbf{version}: Is the same string that in the key of the
                dictionary of ``versions''.
            \item \textbf{description}: The description in this version of the
                package.
            \item \textbf{main}: The Javascript file use as the entry of the program.
                If it is a library, this field is not present.
            \item \textbf{bin}
            \item \textbf{repository}: A JSON dictionary with the information of
                the repository of the proyect:
                \begin{itemize}
                    \item \textbf{type}: A string with the type of the
                        repository. For example ``git''.
                    \item \textbf{url}: The url for cloning the repository.
                \end{itemize}
            \item \textbf{keywords}: A list of keywords to easily find the
                package in the registry.
            \item \textbf{dependencies}: A JSON dictionary where each key if the
                name of the dependency and each value is a semver version (with
                the node's style) or a git URL.
            \item \textbf{devDependencies}: Dependencies needed to run tests or
                in general modify the library.
            \item \textbf{author}: A JSON dictionary with the following keys:
                \begin{itemize}
                    \item \textbf{name}: The name of the author.
                    \item \textbf{email}: The email of the author.
                    \item \textbf{url}: The URL of the webpage of the author
                        (optionl).
                \end{itemize}
            \item \textbf{license}: A name of the license of the code in this
                version.
            \item \textbf{engine}: A list of string where each one is a
                constraint related to the version of node required to run the
                program or library. For example: ``node $>=$0.2.0''.
            \item \textbf{scripts}: A JSON dictionary with scripts to be run in
                the livecycle of the software. The scripts relevante to this
                work are described in the scripts section.
            \item \textbf{dist}: A JSON dictionary with information about the
                source code of the pacakge:
                \begin{itemize}
                    \item \textbf{shasum}: The hash of the file to be
                        downloaded.
                    \item \textbf{tarball}: A tgz file with the code of the
                        version.
                \end{itemize}
            \item \textbf{time}: A dictionary with the timestamp of the
                versions.
            \item \textbf{readme}: The readme to be show to the user in the
                webpage of the NPM registry.
        \end{itemize}
\end{itemize}



\subsection{NPM Scripts}


NPM defines several scripts to be executed in different steps of the instalation
of the software. These are defined in
\footnote{https://docs.npmjs.com/misc/scripts} and the relevant to this work
are:

\begin{itemize}
    \item preinstall
    \item install, postinstall
    \item preuninstall
    \item uninstall
    \item postuninstall
\end{itemize}

A simple map can be made betweet some of this scripts and the one in opam:



\begin{tabularx}{\textwidth}{|c|X|}
        \hline
        NPM script              & opam field    \\ \hline \hline
        preinstall              & build         \\ \hline
        install,postinstall     & install       \\ \hline
        preuninstall,uninstall  & remove        \\ \hline
        \hline
\end{tabularx}



\section{Tools developed}

Several tools were developed to convert the data in the registry into a CUDF
file. Two CouchDB's views, a typescript program and a python program.

The order of execute is: install the views, run deps\_generator.js and finally
npm2cudf.py.

\subsection{Views}

``install\_views.ts`` can install the views directly into CouchDB. The advantage
of using views is that when a new package is added to the registry 

\subsubsection{versions.ts}

This view takes every document from the database and apply a map function to fix
and extract the avalaible versions and license of the packages and numerate
them. For example, a package with the following json description:

\begin{minted}{json}
{
   "_id": "0.workspace",
   "_rev": "19-511dee174891b0ce3b37e11972eda5da",
   "name": "0.workspace",
   "description": "**Status: DEV**",
   "dist-tags": {
       "latest": "0.1.1"
   },
   "versions": {
       "0.0.0": {
       },
       "0.0.1": {
       },
       "0.0.2": {
       }
   }
}
\end{minted}

Will generate this document:

\begin{minted}{json}
[
  {version: "0.0.0", license: "not defined", number: 1},
  {version: "0.0.1", license: "not defined", number: 2}, 
  {version: "0.0.2", license: "not defined", number: 3}
]
\end{minted}

This view also fixes some versions with a bad syntax. Some packages have errors
in the versions, for example, ``1.0.2beta'' with the module ``npm-package-arg''
can fix.

In order to be able to use this module inside CouchDB, we had to create a big
JavsScript file using browserify to merge all the code in only one source file.

This view will be installer in ``registry/\_versions''.

\subsubsection{deps.ts}

This view creates a subsets of the documents in ``registry'' in order of having
a easier and small dataset of the dependencies of a package in each version.

For a package in the registry, generates a json dictionary where each key is a
version and each value is another dictionary with the dependencies. For example,
from the following package:

\begin{minted}{json}
{
   "_id": "0.workspace",
   "_rev": "19-511dee174891b0ce3b37e11972eda5da",
   "name": "0.workspace",
   "description": "**Status: DEV**",
   "dist-tags": {
       "latest": "0.1.1"
   },
   "versions": {
       "0.0.0": {
       },
       "0.0.1": {
         "dependencies": {
           "bash.origin": "^0.1.26"
         }
       },
       "0.0.2": {
         "dependencies": {
           "bash.origin": "^0.1.26"
         }
       }
   }
}
\end{minted}

This view will generate this:

\begin{minted}{json}
{
  "0.0.0": [],
  "0.0.1": {"bash.origin": "^0.1.26"},
  "0.0.2": {"bash.origin": "^0.1.26"}
}
\end{minted}


\subsection{Dependencies generator}

``deps\_generator.ts'' creates a new database where each dependency of a package
is replaced by the versions that can satisfy that constraint.

For example, from the following json:

\begin{minted}{json}
{
   "_id": "0.workspace",
   "_rev": "19-511dee174891b0ce3b37e11972eda5da",
   "name": "0.workspace",
   "description": "**Status: DEV**",
   "dist-tags": {
       "latest": "0.1.1"
   },
   "versions": {
       "0.0.0": {
       },
       "0.0.1": {
         "dependencies": {
           "bash.origin": "^0.1.26"
         }
       },
       "0.0.2": {
         "dependencies": {
           "bash.origin": "^0.1.26"
         }
       }
   }
}
\end{minted}

We get:

\begin{minted}{json}
{
   "_id": "0.workspace",
   "_rev": "1-d7b59e993b1528d279774d3eb8e62969",
   "value": {
       "0.0.0": {
       },
       "0.0.1": {
           "bash.origin": [
               "0.1.26",
               "0.1.27",
               "0.1.28",
               "0.1.29",
               "0.1.30",
               "0.1.31"
           ]
       },
       "0.0.2": {
           "bash.origin": [
               "0.1.26",
               "0.1.27",
               "0.1.28",
               "0.1.29",
               "0.1.30",
               "0.1.31"
           ]
       },
       "0.0.3": {
           "bash.origin": [
               "0.1.26",
               "0.1.27",
               "0.1.28",
               "0.1.29",
               "0.1.30",
               "0.1.31"
           ]
       }
  }
}
\end{minted}

This processing is done in batches of 5000 packages to prevent out-of-memory
execptions.

For the range evaluation we use the semver module
\footnote{https://github.com/npm/node-semver} and a cache to speedup the
calculation of the ranges.


All the generated json are stored in a second database called
``registry\_dependencies''.



\subsection{CUDF file generator}

The generation of the CUDF file in done in npm2cudf.py, a python program that
takes the data generated in the views and in ``deps\_generator.ts''.

This program iterates over all the packages in ``registry\_dependencies'' and
for each version it generates a CUDF entry with the dependencies tranformed into
numbers using the information generated in the view ``\_versions''.

The end result is a text file of about 600MB that can be used by the tools in
Dose to analyse the packages in NPM. However, to analize this dataset it is
required to run the tools in a computer with at least 64 GB of ram for
distcheck. In a computer with an Intel Core i5 running at 3GHz it takes about 21
hours to execute distcheck.


\section{Additions to Dose}

A few additions have been done in dose with the help of Pietro Abate in the
parsing of the semantic versioning and the range syntax of NPM.

NPM uses semantic versioning for the definition of the versions and a custom
syntax for defining ranges.

The semantic versioning of dose was expanded to include the comparing of the
prerelease section. For example, a semantic version can be ``1.2.3-pre.0''. A
simple comparision is done in the first three componentes of the version but in
the last one (pre.0), the preelease comparision is done as follows:

\begin{itemize}
  \item Split the string in the dots.
  \item For each one of the elements, if it is a number, compare using the
    numeric comparator.
  \item If it is a string, compare using the string comparator.
  \item The version with less elements is bigger than a version with less.
\end{itemize}


A range in npm is defined by the following grammar:

\begin{verbatim}
range-set  ::= range ( logical-or range ) *
logical-or ::= ( ' ' ) * '||' ( ' ' ) *
range      ::= hyphen | simple ( ' ' simple ) * | ''
hyphen     ::= partial ' - ' partial
simple     ::= primitive | partial | tilde | caret
primitive  ::= ( '<' | '>' | '>=' | '<=' | '=' | ) partial
partial    ::= xr ( '.' xr ( '.' xr qualifier ? )? )?
xr         ::= 'x' | 'X' | '*' | nr
nr         ::= '0' | ['1'-'9'] ( ['0'-'9'] ) *
tilde      ::= '~' partial
caret      ::= '^' partial
qualifier  ::= ( '-' pre )? ( '+' build )?
pre        ::= parts
build      ::= parts
parts      ::= part ( '.' part ) *
part       ::= nr | [-0-9A-Za-z]+
\end{verbatim}

with a lot of syntax sugar. The desugaring process is implemented in the npm
module of dose.




\section{ONPM and Opam Repository}

\subsection{Opam Repository}

An opam repository have been done
\footnote{https://github.com/fgaray/opam-npm-repository} that contains the NPM
packages converted into a format that can be understand by opam. A simple
example can be found \footnote{https://github.com/fgaray/opam-npm-repository/tree/master/packages/underscore/underscore.1.8.3}
there for the underscore JavaScript library.

This folder contains three files and a folder:

\begin{itemize}
  \item \textbf{descr}: A simple description for the package.
  \item \textbf{opam}: Information of the package including dependencies.
  \item \textbf{url}: Address of the package and checksum.
  \item \textbf{files}: Folder with just one file called underscore.install with
    routes of where to put the files from underscore.
\end{itemize}

In the case of underscore, the install file contains:

\begin{verbatim}
lib: [
    "underscore.js" 
    "underscore-min.js"
    "package.json"
    "underscore-min.map"
]
\end{verbatim}

That tells opam to copy this files into the lib directory.


\subsection{Node's module system}

Node is the responsable of the search of modules in the file system with a
function called ``require''. The algorithm
\footnote{https://nodejs.org/docs/latest/api/modules.html\#modules\_all\_together}
works testing each directory in an internal list of node and in each one of this
node try to find a folder with the name of the module requested.

For example, if we want to load ``underscore'', then require("underscore") in a
list of ``\[X, Y, Z\]'' directories will try to load the directories
``X/underscore'', ``Y/underscore'' and ``Z/underscore''.

We can use the enviroment variable \$NODE\_PATH to add new directories to be
searched by NPM.

\subsection{ONPM}

This program uses the command line of opam to install packages from the
opam-npm-repository creating the needed enviroment variables and switchs in
opaswitchs in opam.

There is a command implemented called ``install'' that can receive as an
argument a package name to be installed in the global-js switch.

If install is called without argument, then it searchs for a package.json file
in the current directory, gets all the dependencies and install them in a new
switch with the name of the proyect.


\end{document}
